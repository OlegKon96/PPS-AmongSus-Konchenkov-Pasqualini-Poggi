\section*{Introduzione}

Il progetto si pone come obiettivo la realizzazione del videogioco Among Us, piattaforma multiplayer che supporta da 4 fino a 10 giocatori. Durante ogni partita, da 1 a 3 giocatori vengono selezionati casualmente e contrassegnati come \textit{Impostor}, mentre gli altri saranno \textit{Crewmate}. Gli \textit{Impostor} avranno la capacit\`a di sabotare i sistemi della mappa, attraversare le "botole" presenti in alcuni punti della mappa per passare da una stanza all'altra pi\`u rapidamente, identificare qualsiasi altro \textit{Impostor} ed il loro scopo sarà quello di uccidere i \textit{Crewmate}.\\
L'obiettivo dei \textit{Crewmate} \`e scoprire gli \textit{Impostor} ed eliminarli prima di essere assassinati o sabotati. Quando un giocatore muore, diventa un fantasma, il cui obiettivo è aiutare i membri rimanenti della propria squadra.\\
Se un giocatore trova un cadavere, pu\`o segnalarlo, il che porter\`a ad una riunione di gruppo in cui i giocatori discuteranno su chi pensano sia un \textit{Impostor}, sulla base delle prove che hanno visto vicino l'omicidio. Se si raggiunge la maggioranza dei voti, la persona prescelta viene espulsa dalla mappa e verr\`a rivelato se era un \textit{Impostor} o meno. I giocatori possono chiedere in qualsiasi momento, ma solo una volta a partita, un \textit{"Emergency Meeting"} dirigendosi verso l'inizio della mappa ed attivando il bottone apposito.